\documentclass{article}
\usepackage{amsmath}
\usepackage{graphicx}
\usepackage{hyperref}
\usepackage{fancyhdr}
\usepackage{xcolor}
\usepackage{listings}

% Define custom colors
\definecolor{bg}{rgb}{0.64, 0.64, 0.82}
\definecolor{frame}{rgb}{0.59, 0.47, 0.71}
\definecolor{keyword}{rgb}{0.63, 0.14, 0.15}
\definecolor{comment}{rgb}{0.44, 0.5, 0.56}
\definecolor{string}{rgb}{0.12, 0.56, 0.18}

% Set custom listings options
\lstset{
    backgroundcolor=\color{bg},
    frame=single,
    rulecolor=\color{frame},
    basicstyle=\ttfamily\small,
    keywordstyle=\color{keyword}\bfseries,
    commentstyle=\color{comment},
    stringstyle=\color{string},
    showstringspaces=false,
    breaklines=true,
    xleftmargin=2mm,
    xrightmargin=2mm
}

\pagestyle{fancy}
\fancyhf{}
\fancyhead[L]{MAD 100 - Java Programming I - Fall 2024}
\fancyhead[R]{Instructor: Franco Iacobacci \thepage}
\title{Java Programming 1 - Week 11 Notes}
\author{Hia Al Saleh}
\date{November 14th, 2024}

\begin{document}

\maketitle
\tableofcontents
\newpage 

\section{Introduction}
In this week's lecture, we are diving deeper into Object-Oriented Programming (OOP), focusing on building super and subclasses, inheritance, and method overriding and overloading. We will use these concepts to create a game in which each character belongs to a specific race (\texttt{Orc}, \texttt{Elf}, \texttt{Human}) and has unique abilities and attributes.

\section{Overview of Super and Subclasses}
A \textbf{superclass} (also called a \textbf{base} or \textbf{parent class}) is a general class that defines common properties and methods for all its subclasses. A \textbf{subclass} (also known as a \textbf{derived} or \textbf{child class}) inherits the properties and methods from the superclass and can add or override them.

\textbf{Example:}
\begin{lstlisting}[language=Java]
class Character {
    // Common attributes for all characters
    String name;
    int age;
    int health;
    
    public Character(String name, int age) {
        this.name = name;
        this.age = age;
        this.health = 100; // default health
    }
    
    public void displayInfo() {
        System.out.println("Name: " + name + ", Age: " + age + ", Health: " + health);
    }
}

class Orc extends Character {
    int strength;

    public Orc(String name, int age) {
        super(name, age);
        this.strength = 120; // default orc strength
    }
    
    public void displayInfo() {
        super.displayInfo();
        System.out.println("Strength: " + strength);
    }
}
\end{lstlisting}

In this example, \texttt{Orc} is a subclass that inherits from the \texttt{Character} superclass, adding a specific attribute (\texttt{strength}) and overriding the \texttt{displayInfo()} method to include it.

\section{Creating the \texttt{Character} Superclass}
The \texttt{Character} superclass defines attributes and behaviors that all characters will share, such as name, age, health, and methods to access these attributes. We also define an abstract \texttt{communicate()} method, which each subclass will implement differently.

\subsection{Attributes}
Each \texttt{Character} will have the following attributes:
\begin{itemize}
    \item \textbf{Name:} The character’s name.
    \item \textbf{Age:} The character’s age.
    \item \textbf{Health:} An integer representing health points.
    \item Other attributes include intelligence, strength, defense, and agility.
\end{itemize}

\subsection{Constructor and Methods}
The constructor initializes name and age, while getters and setters allow access to other private attributes.

\textbf{Example:}
\begin{lstlisting}[language=Java]
public abstract class Character {
    private String name;
    private int age;
    private int health;

    public Character(String name, int age) {
        this.name = name;
        this.age = age;
        this.health = 100; // Default health
    }

    public String getName() {
        return name;
    }

    public int getAge() {
        return age;
    }

    public int getHealth() {
        return health;
    }

    // Abstract method to be implemented by subclasses
    public abstract void communicate();
}
\end{lstlisting}

\section{Creating Subclasses for Different Races}
The subclasses (\texttt{Orc}, \texttt{Elf}, and \texttt{Human}) extend \texttt{Character} and provide specific implementations of the \texttt{communicate()} method and race-specific attributes.

\subsection{Orc Subclass}
The \texttt{Orc} class represents a strong character type with higher-than-average strength but lower intelligence.

\textbf{Example:}
\begin{lstlisting}[language=Java]
public class Orc extends Character {
    private int strength;

    public Orc(String name, int age) {
        super(name, age);
        this.strength = 150;
    }

    @Override
    public void communicate() {
        System.out.println("Grunts and roars");
    }
}
\end{lstlisting}

\subsection{Elf Subclass}
The \texttt{Elf} class has higher agility and intelligence, suitable for stealth and magic.

\textbf{Example:}
\begin{lstlisting}[language=Java]
public class Elf extends Character {
    private int agility;
    
    public Elf(String name, int age) {
        super(name, age);
        this.agility = 120;
    }

    @Override
    public void communicate() {
        System.out.println("Speaks in an elegant, mystical language");
    }
}
\end{lstlisting}

\subsection{Human Subclass}
The \texttt{Human} class is well-balanced across attributes.

\textbf{Example:}
\begin{lstlisting}[language=Java]
public class Human extends Character {
    private int wisdom;

    public Human(String name, int age) {
        super(name, age);
        this.wisdom = 100;
    }

    @Override
    public void communicate() {
        System.out.println("Speaks in a common language");
    }
}
\end{lstlisting}

\section{Creating Arrays of Characters}
In Java, we can create arrays of objects to store instances of different subclasses. This allows us to perform actions on each object in a unified way.

\textbf{Example:}
\begin{lstlisting}[language=Java]
public class Game {
    public static void main(String[] args) {
        Character[] characters = {
            new Orc("Gor", 30),
            new Elf("Aelar", 100),
            new Human("John", 25)
        };

        for (Character character : characters) {
            character.displayInfo();
            character.communicate();
        }
    }
}
\end{lstlisting}

This array contains different types of \texttt{Character} objects (an \texttt{Orc}, an \texttt{Elf}, and a \texttt{Human}), and each one can use the \texttt{communicate()} method, demonstrating polymorphism.

\section{Abstract Classes and Methods}
An \textbf{abstract class} cannot be instantiated and is meant to be a base class for other classes. In our game, the \texttt{Character} class is abstract to ensure no character is created without a specific race.

\textbf{Abstract Method Example:}
\begin{lstlisting}[language=Java]
public abstract class Character {
    public abstract void communicate();
}
\end{lstlisting}

\textbf{Polymorphism in Action:}
In the \texttt{Game} class example, polymorphism allows each character type to execute its specific \texttt{communicate()} method when called on the array of \texttt{Character} objects.

\section{Homework and Test Preparation}
\begin{itemize}
    \item Review Chapter 11 of the textbook, focusing on abstract classes and polymorphism.
    \item Be prepared to demonstrate how inheritance and method overriding work in the context of the game characters.
    \item Experiment with adding more races, such as \texttt{Werewolf} or \texttt{Vampire}, each implementing unique \texttt{communicate()} behaviors.
\end{itemize}

\section{Next Week}
Prepare for a test covering:
\begin{itemize}
    \item Creating classes, subclasses, and inheritance.
    \item Method overriding and overloading.
    \item Abstract classes and polymorphism.
\end{itemize}

\end{document}