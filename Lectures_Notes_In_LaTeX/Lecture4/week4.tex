\documentclass{article}
\usepackage{amsmath}
\usepackage{graphicx}
\usepackage{hyperref}
\usepackage{fancyhdr}
\usepackage{xcolor}
\usepackage{listings}

% Define custom colors
\definecolor{bg}{rgb}{0.64, 0.64, 0.82}
\definecolor{frame}{rgb}{0.59, 0.47, 0.71}
\definecolor{keyword}{rgb}{0.63, 0.14, 0.15}
\definecolor{comment}{rgb}{0.44, 0.5, 0.56}
\definecolor{string}{rgb}{0.12, 0.56, 0.18}

% Set custom listings options
\lstset{
    backgroundcolor=\color{bg},
    frame=single,
    rulecolor=\color{frame},
    basicstyle=\ttfamily\small,
    keywordstyle=\color{keyword}\bfseries,
    commentstyle=\color{comment},
    stringstyle=\color{string},
    showstringspaces=false,
    breaklines=true,
    xleftmargin=2mm,
    xrightmargin=2mm
}

\pagestyle{fancy}
\fancyhf{}
\fancyhead[L]{MAD 100 - Java Programming I - Fall 2024}
\fancyhead[R]{Instructor: Franco Iacobacci \thepage}
\title{Java Programming Notes}
\author{Hia Al Saleh}
\date{September 26th, 2024}

\begin{document}
\maketitle
\tableofcontents
\newpage

\section{Repetition Structures}

\subsection{Topics}
\begin{itemize}
    \item  Sentinel values
    \item  While Loop
    \item  For Loop
    \item  Do While Loop
    \item  Nested Loops
    \item  Boolean Conditions
\end{itemize}


\subsection{Creating a new project in IntelliJ}
Steps to create a new Java project:
\begin{enumerate}
    \item Open IntelliJ and select \texttt{File > New > Project}.
    \item Select \texttt{Java Project} and choose \texttt{SDK 1.8}.
    \item Name the project \textbf{RandomGame}.
\end{enumerate}


\subsection{Generating Random Numbers}
To generate random numbers in Java:
\begin{itemize}
    \item Create a random object using \texttt{Random()}.
    \item Seed the random object using the current time in milliseconds.
    \item Use \texttt{random.nextInt(n)} to generate numbers between 0 and n-1.
\end{itemize}

\begin{lstlisting}[language=java]
import java.util.Random;
import java.util.Scanner;

public class EasterEggGame {
    public static void main(String[] args) {
        final Boolean DEBUG = false;
        Scanner input = new Scanner(System.in);
        Random random = new Random();
        int totalScore = 0;
        int difficulty;
        //Welcome user to the game and ask them for difficulty level
        System.out.println("Welcome to the ultra secret guessing game.");
        do {
            System.out.println("Please select a difficulty\n" +
                    "1) Easy (1-10)\n" +
                    "2) Medium (1-25)\n" +
                    "3) Hard (1-50)\n");
            //Accept user input for difficulty level

             difficulty = input.nextInt();
        } while (difficulty > 3 );
        //Ask the user the number of rounds, take as input
        System.out.println("How many rounds would you like to play?");
        int numOfRounds = input.nextInt();
        // PER ROUND:
        for (int i =0; i< numOfRounds; i++){
            int winningNumber=0;
            //  Generate proper random answer based on difficulty level
            switch (difficulty){
                case 1: winningNumber = random.nextInt(10)+1;
                        break;
                case 2: winningNumber = random.nextInt(25)+1;
                        break;
                case 3: winningNumber = random.nextInt(50)+1;
                        break;
            }
            //  Ask user for number guess until correct
            int userChoice;
            int guesses=0;
            if(DEBUG) {
                System.out.println("DEBUG - " + winningNumber);
            }
            do{
                System.out.println("Please enter your guess: ");
                userChoice = input.nextInt();
                guesses++;
            } while (userChoice!=winningNumber);
            //  Add score to total score
            totalScore+=guesses;
            System.out.println("It took "+guesses+" to guess the number");
        }

        //Display Total Score (lower is better)
        System.out.println("Total score: "+totalScore);
    }
}
\end{lstlisting}

\subsection{Repetition Structures}
Three main types of loops:
\begin{itemize}
    \item \textbf{For Loop}: Used for executing code a set number of times.
    \item \textbf{While Loop}: Executes code while a condition is true.
    \item \textbf{Do While Loop}: Similar to the while loop, but guarantees execution at least once.
\end{itemize}
\begin{lstlisting}[language=java]
    public class DoWhile {
    public static void main(String[] args) {
        int number = 10;
        do{
            System.out.println("Do While " + number);
        }while (number < 9);

        while (number <9){
            System.out.println("while "+number);
        }
        /*
        while (number < 20){
            System.out.println("while "+number);
            number++;
        }*/
    }
}
\end{lstlisting}

\subsection{Boolean Conditions}
\begin{itemize}
    \item Loops often rely on Boolean conditions to determine when to stop executing.
    \item Boolean expressions return either \texttt{true} or \texttt{false}.
\end{itemize}
\section{TestRandom}
\begin{lstlisting}[language=java]
import java.util.Random;

public class TestRandom {
    public static void main(String[] args) {
        Random rand = new Random();//new Random(20);
        System.out.println(rand.nextInt());
        //System.out.println((rand.nextInt(10)+1)); //1-10
        //System.out.println(rand.nextInt(31)+20); //20-50
        for (int i = 0;i<10;i++){
            System.out.println(i+" - "+(rand.nextInt(31)+20)); //20-50
        }
        int randomNum;
        do {
            randomNum = rand.nextInt(31) + 20;
            System.out.println(randomNum);
        } while (randomNum >= 20);
    }
}    
\end{lstlisting}


\subsection*{Homework}
\begin{itemize}
    \item Read Chapter 5 of the textbook.
    \item Test the Random Game program for generating random numbers and tracking user scores.
\end{itemize}

\section*{Next Week}
Topics include:
\begin{itemize}
    \item Repetition structures
    \item Sentinel values
    \item Nested loops
    \item Boolean conditions
    \item String manipulation
\end{itemize}

\end{document}