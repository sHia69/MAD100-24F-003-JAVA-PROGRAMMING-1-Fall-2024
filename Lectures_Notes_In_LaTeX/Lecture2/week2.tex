\documentclass{article}
\usepackage{amsmath}
\usepackage{graphicx}
\usepackage{hyperref}
\usepackage{fancyhdr}
\usepackage{xcolor}
\usepackage{listings}

% Define custom colors
\definecolor{bg}{rgb}{0.64, 0.64, 0.82}
\definecolor{frame}{rgb}{0.59, 0.47, 0.71}
\definecolor{keyword}{rgb}{0.63, 0.14, 0.15}
\definecolor{comment}{rgb}{0.44, 0.5, 0.56}
\definecolor{string}{rgb}{0.12, 0.56, 0.18}

% Set custom listings options
\lstset{
    backgroundcolor=\color{bg},
    frame=single,
    rulecolor=\color{frame},
    basicstyle=\ttfamily\small,
    keywordstyle=\color{keyword}\bfseries,
    commentstyle=\color{comment},
    stringstyle=\color{string},
    showstringspaces=false,
    breaklines=true,
    xleftmargin=2mm,
    xrightmargin=2mm
}

\pagestyle{fancy}
\fancyhf{}
\fancyhead[L]{MAD 100 - Java Programming I - Fall 2024}
\fancyhead[R]{Instructor: Franco Iacobacci \thepage}

\title{Java Programming Notes}
\author{Hia Al Saleh}
\date{September 12th, 2024}

\begin{document}
\maketitle
\tableofcontents
\newpage 
\section{Topics for This Week}
\begin{itemize}
    \item Variables and Console
    \item Assignment Statements
    \item Naming Conventions
    \item Data Types and Operators
    \item Numeric Type Casting
\end{itemize}

\section{Client Request: Truss Goodman}
\textbf{Initial Requirements:}
\begin{itemize}
    \item Input shingle cost, roof size, and installation cost.
    \item Calculate taxes and output the total before and after taxes.
\end{itemize}

\textbf{Steps to Implement:}
\begin{enumerate}
    \item Allow user input for shingle cost.
    \item Allow user input for roof size (in square feet).
    \item Allow user input for installation cost (per square foot).
    \item Add a named constant for relevant taxes.
    \item Calculate totals (before and after taxes).
    \item Output the totals to the console.
\end{enumerate}

\section{Java Input: Scanner Class}
\textbf{Step 1-3:} Using the Scanner class for user input:
\begin{itemize}
    \item Import the Scanner class: \texttt{import java.util.Scanner;}
    \item Create a new scanner object: \texttt{Scanner input = new Scanner(System.in);}
    \item Store input in variables using \texttt{input.nextDouble()} for numeric input.
\end{itemize}

\section{Testing the Program}
\begin{itemize}
    \item Compile and run the program.
    \item Input values for shingle cost, roof size, and installation cost.
    \item Verify output before and after taxes.
\end{itemize}

\section{Variable Naming Conventions}
\textbf{Camel Case:}
\begin{itemize}
    \item Variables should be named using camel case, e.g., \texttt{thisIsCamelCase}.
\end{itemize}

\textbf{Named Constants:}
\begin{itemize}
    \item Constants should be named in uppercase.
    \item Example: \texttt{final double HST = 1.13;}
\end{itemize}

\section{Final Output and Client Feedback}
\begin{itemize}
    \item The final output includes totals before and after taxes.
    \item Potential future changes for client needs.
\end{itemize}

\section{Creating a New Java Project: SimpleCalculator}
\textbf{Steps:}
\begin{enumerate}
    \item Create a new project in IntelliJ with SDK 1.8.
    \item Create a new Java class: \texttt{SimpleCalculator}.
    \item Use \texttt{System.out.println()} to display options.
\end{enumerate}

\textbf{Example Code:}
\begin{lstlisting}[language=Java]
System.out.println("Welcome to the calculator app");
System.out.println("1) Perform Multiplication");
System.out.println("2) Perform Division");
System.out.println("3) Perform Subtraction");
System.out.println("4) Perform Addition");
System.out.println("5) Perform Modulus");
System.out.println("Please Select A Number:");
\end{lstlisting}

\section{Error Types}
\begin{itemize}
    \item \textbf{Syntax Errors:} Errors in code structure (e.g., missing braces).
    \item \textbf{Logic Errors:} Code executes but produces incorrect results.
    \item \textbf{Runtime Errors:} Occur when unexpected input is provided.
\end{itemize}

\section{Homework}
\begin{itemize}
    \item Read pages 120-148 of the textbook.
\end{itemize}

\section{Next Week's Topics}
\begin{itemize}
    \item Boolean Data Types
    \item Selection Structures (if, if-else statements)
    \item Boolean Conditions
    \item Relational Operators
    \item Math Library - Random Numbers
    \item Logic Operators
    \item Conditional Expressions
\end{itemize}

\end{document}