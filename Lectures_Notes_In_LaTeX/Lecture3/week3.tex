\documentclass{article}
\usepackage{amsmath}
\usepackage{graphicx}
\usepackage{hyperref}
\usepackage{fancyhdr}
\usepackage{xcolor}
\usepackage{listings}

% Define custom colors
\definecolor{bg}{rgb}{0.64, 0.64, 0.82}
\definecolor{frame}{rgb}{0.59, 0.47, 0.71}
\definecolor{keyword}{rgb}{0.63, 0.14, 0.15}
\definecolor{comment}{rgb}{0.44, 0.5, 0.56}
\definecolor{string}{rgb}{0.12, 0.56, 0.18}

% Set custom listings options
\lstset{
    backgroundcolor=\color{bg},
    frame=single,
    rulecolor=\color{frame},
    basicstyle=\ttfamily\small,
    keywordstyle=\color{keyword}\bfseries,
    commentstyle=\color{comment},
    stringstyle=\color{string},
    showstringspaces=false,
    breaklines=true,
    xleftmargin=2mm,
    xrightmargin=2mm
}

\pagestyle{fancy}
\fancyhf{}
\fancyhead[L]{MAD 100 - Java Programming I - Fall 2024}
\fancyhead[R]{Instructor: Franco Iacobacci \thepage}

\title{Java Programming Notes}
\author{Hia Al Saleh}
\date{September 19th, 2024}

\begin{document}
\maketitle
\tableofcontents
\newpage

\section{Math Library in Java}
The Java Math library offers various functions to perform mathematical operations, such as calculating exponents, square roots, and logarithms.

\subsection{Key Math Functions}
Here are some common functions from the Math library:
\begin{itemize}
    \item \texttt{Math.pow(base, exponent)}: Calculates the value of the base raised to the power of the exponent.
    \item \texttt{Math.sqrt(number)}: Returns the square root of the given number.
    \item \texttt{Math.log10(number)}: Returns the base 10 logarithm of the given number.
\end{itemize}

\subsection{Example: Exponent Calculation}
Here is an example that calculates an exponent using \texttt{Math.pow()}:
\begin{lstlisting}[language=Java]
double base = 2;
double exponent = 3;
double result = Math.pow(base, exponent);
System.out.println("Result: " + result);
\end{lstlisting}

\section{Conditional Logic in Java}
Conditional logic in Java allows the program to make decisions based on certain conditions. The most common conditional structures are the \texttt{if}, \texttt{else if}, and \texttt{else} statements.

\subsection{If Statement}
The \texttt{if} statement executes a block of code if the given condition evaluates to true.

\begin{lstlisting}[language=Java]
int a = 5;
if (a > 3) {
    System.out.println("a is greater than 3");
}
\end{lstlisting}

\subsection{If-Else Statement}
The \texttt{if-else} statement provides an alternative block of code to execute if the condition evaluates to false.

\begin{lstlisting}[language=Java]
int b = 2;
if (b > 3) {
    System.out.println("b is greater than 3");
} else {
    System.out.println("b is less than or equal to 3");
}
\end{lstlisting}

\subsection{Else-If Statement}
The \texttt{else-if} statement allows multiple conditions to be checked in sequence. If the first condition fails, the program checks the next one.

\begin{lstlisting}[language=Java]
int c = 7;
if (c > 10) {
    System.out.println("c is greater than 10");
} else if (c > 5) {
    System.out.println("c is greater than 5 but less than or equal to 10");
} else {
    System.out.println("c is less than or equal to 5");
}
\end{lstlisting}

\section{Boolean Data Type}
In Java, the \texttt{boolean} data type represents two possible values: \texttt{true} or \texttt{false}. Boolean values are commonly used in conditional expressions.

\begin{lstlisting}[language=Java]
boolean isHungry = true;
if (isHungry) {
    System.out.println("I am hungry");
} else {
    System.out.println("I am not hungry");
}
\end{lstlisting}

\section{Comparison Operators}
Comparison operators are used to compare two values. These operators return a boolean value (\texttt{true} or \texttt{false}).

\begin{itemize}
    \item \texttt{==}: Equal to
    \item \texttt{!=}: Not equal to
    \item \texttt{>}: Greater than
    \item \texttt{<}: Less than
    \item \texttt{>=}: Greater than or equal to
    \item \texttt{<=}: Less than or equal to
\end{itemize}

\subsection{Example: Using Comparison Operators}
\begin{lstlisting}[language=Java]
int x = 10;
int y = 5;
System.out.println(x > y); // true
System.out.println(x == y); // false
System.out.println(x != y); // true
\end{lstlisting}

\section{Logical Operators}
Logical operators allow combining multiple boolean expressions into a single expression.

\begin{itemize}
    \item \texttt{\&\&}: Logical AND (both conditions must be true)
    \item \texttt{||}: Logical OR (at least one condition must be true)
    \item \texttt{!}: Logical NOT (inverts the boolean value)
\end{itemize}

\subsection{Example: Logical Operators}
\begin{lstlisting}[language=Java]
boolean isRaining = false;
boolean haveUmbrella = true;

if (!isRaining || haveUmbrella) {
    System.out.println("I can go outside");
}
\end{lstlisting}

\section{Creating a Java Class in IntelliJ}
To create a new Java class in IntelliJ, follow these steps:
\begin{enumerate}
    \item Open IntelliJ and select \texttt{File > New > Project}.
    \item Choose Java and set the project SDK to version 1.8.
    \item Give the project a name, e.g., \texttt{FunctionCalculator}.
    \item Right-click on the \texttt{src} folder and select \texttt{New > Java Class}.
    \item Name the class \texttt{FunctionCalculator} and hit enter.
\end{enumerate}

\subsection{Writing a Program}
In the new \texttt{FunctionCalculator.java} file, you can start by creating the \texttt{main} method using the following command:
\begin{lstlisting}[language=Java]
public class FunctionCalculator {
    public static void main(String[] args) {
        System.out.println("Welcome to Function Calculator!");
    }
}
\end{lstlisting}

\section{Homework and Next Steps}
Read Chapter 3 and 4 of your textbook to prepare for next week's topics, which will cover:
\begin{itemize}
    \item Repetition Structures
    \item Sentinel Values
    \item While and For Loops
    \item Nested Loops
\end{itemize}

\end{document}