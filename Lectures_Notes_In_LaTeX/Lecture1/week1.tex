\documentclass{article}
\usepackage{amsmath}
\usepackage{graphicx}
\usepackage{hyperref}
\usepackage{fancyhdr}
\usepackage{xcolor}
\usepackage{listings}

% Define custom colors
\definecolor{bg}{rgb}{0.64, 0.64, 0.82}
\definecolor{frame}{rgb}{0.59, 0.47, 0.71}
\definecolor{keyword}{rgb}{0.63, 0.14, 0.15}
\definecolor{comment}{rgb}{0.44, 0.5, 0.56}
\definecolor{string}{rgb}{0.12, 0.56, 0.18}

% Set custom listings options
\lstset{
    backgroundcolor=\color{bg},
    frame=single,
    rulecolor=\color{frame},
    basicstyle=\ttfamily\small,
    keywordstyle=\color{keyword}\bfseries,
    commentstyle=\color{comment},
    stringstyle=\color{string},
    showstringspaces=false,
    breaklines=true,
    xleftmargin=2mm,
    xrightmargin=2mm
}

\pagestyle{fancy}
\fancyhf{}
\fancyhead[L]{MAD 100 - Java Programming I - Fall 2024}
\fancyhead[R]{Instructor: Franco Iacobacci \thepage}
\title{Java Programming 1 - Week 1 Notes}
\author{Hia Al Saleh}
\date{September 5th, 2024}

\begin{document}
\maketitle
\tableofcontents
\newpage
\section{Topics for this Week}
\begin{itemize}
    \item Computers
    \item Operating Systems
    \item Programming
    \item Java
    \item IDEs
    \item Hello World
\end{itemize}

\section{History}
\textbf{What is a Computer?}
\begin{itemize}
    \item A computer is a general-purpose device comprised of hardware.
    \item Computer hardware can execute tasks via programming.
    \item Programs, or software, are written using code.
\end{itemize}

\textbf{Operating System (OS):}
\begin{itemize}
    \item The OS is the largest software component, controlling and managing hardware.
    \item It allocates hardware resources.
\end{itemize}

\section{What is Programming?}
\begin{itemize}
    \item Programming is the process of writing software.
    \item There are hundreds of programming languages, each suited to different tasks.
    \item In this course, we focus on \textbf{Java}.
    \item Java is widely used for building desktop, server, and Android mobile applications.
\end{itemize}

\section{Language Overview}
\begin{tabular}{|l|l|}
\hline
\textbf{Language} & \textbf{Primary Use} \\
\hline
C & Low-level operations, OS development \\
Python & General-purpose, used in math, science \\
PHP & Server-side web development \\
C++ & General-purpose, used for applications, games \\
Java & General-purpose, used on all OS and mobile platforms \\
\hline
\end{tabular}

\section{Java Programming Language}
\textbf{What is Java?}
\begin{itemize}
    \item Java was developed by Sun Microsystems, led by James Gosling.
    \item It is platform-independent: \emph{Write once, run anywhere}.
    \item Java code is compiled into \textbf{bytecode}, which runs on the \textbf{Java Virtual Machine (JVM)}.
    \item JVM allows Java bytecode to run on any operating system (Windows, Mac, Linux).
\end{itemize}

\textbf{Java Compilation Process:}
\begin{itemize}
    \item High-level code (Java) is compiled into bytecode.
    \item The JVM interprets bytecode and translates it for the specific OS.
\end{itemize}

\section{Hello World in Java}
\subsection*{Steps to Build the First Java Program}
\begin{enumerate}
    \item Open \textbf{IntelliJ IDEA} IDE.
    \item Create a new Java project (\texttt{File > New > Project}).
    \item Choose the \textbf{Java SDK} (version 15).
    \item Name the project \emph{HelloWorld}.
    \item Inside the \texttt{src} directory, create a new Java class named \emph{HelloWorld}.
    \item Inside the class, type \texttt{psvm} and hit Enter for auto-completion of the main method.
\end{enumerate}

\textbf{HelloWorld.java File Structure:}
\begin{lstlisting}[language=java]
public class HelloWorld {
    public static void main(String[] args) {
        System.out.println("Hello World");
    }
}
\end{lstlisting}

\subsection*{Compiling and Running the Program}
\begin{itemize}
    \item To compile and run: \texttt{Run > Run > HelloWorld}.
    \item The output will be displayed in the console.
\end{itemize}

\section{Arithmetic and Operators in Java}
\textbf{Example Code:}
\begin{lstlisting}
public class RoofQuote {
    public static void main(String[] args) {
        System.out.println((5 * 10) / 2);  // Outputs: 25
    }
}
\end{lstlisting}
Java follows the \textbf{BEDMAS} rules:
\begin{itemize}
    \item Brackets, Exponents, Division/Multiplication (left to right), Addition/Subtraction (left to right).
\end{itemize}

\section{Variables in Java}
\textbf{What are Variables?}
\begin{itemize}
    \item Variables store data for later use and have specific \textbf{data types}.
    \item Java is a \emph{strongly typed} language, meaning variable types must be declared.
\end{itemize}

\textbf{Variable Examples:}
\begin{lstlisting}[language=java]
int age = 45;
String name = "Truss Goodman";
double bankAccount = 43000.92;
\end{lstlisting}

\textbf{Exercise:}
\begin{itemize}
    \item Create two variables (\texttt{number1}, \texttt{number2}).
    \item Output their sum and product.
\end{itemize}
\newpage 

\section{Homework \& Next Week}
\begin{itemize}
    \item Read pages 1-36 of your textbook.
    \item Variables in depth
    \item Reading input from the console
    \item Assignment statements \& expressions
    \item Naming conventions
    \item Data types
    \item Operators \& type conversions
\end{itemize}

\end{document}