\documentclass{article}
\usepackage{amsmath}
\usepackage{graphicx}
\usepackage{hyperref}
\usepackage{fancyhdr}
\usepackage{xcolor}
\usepackage{listings}

% Define custom colors
\definecolor{bg}{rgb}{0.64, 0.64, 0.82}
\definecolor{frame}{rgb}{0.59, 0.47, 0.71}
\definecolor{keyword}{rgb}{0.63, 0.14, 0.15}
\definecolor{comment}{rgb}{0.44, 0.5, 0.56}
\definecolor{string}{rgb}{0.12, 0.56, 0.18}

% Set custom listings options
\lstset{
    backgroundcolor=\color{bg},
    frame=single,
    rulecolor=\color{frame},
    basicstyle=\ttfamily\small,
    keywordstyle=\color{keyword}\bfseries,
    commentstyle=\color{comment},
    stringstyle=\color{string},
    showstringspaces=false,
    breaklines=true,
    xleftmargin=2mm,
    xrightmargin=2mm
}

\pagestyle{fancy}
\fancyhf{}
\fancyhead[L]{MAD 100 - Java Programming I - Fall 2024}
\fancyhead[R]{Instructor: Franco Iacobacci \thepage}
\title{Java Programming 1 - Week 8 Notes}
\author{Hia Al Saleh}
\date{October 24th, 2024}

\begin{document}

\maketitle
\tableofcontents
\newpage 

\section{SpyTek Caesar Cipher}
In this week's lesson, we covered topics related to encryption and decryption using the Caesar Cipher. We focused on the following core concepts:
\begin{itemize}
    \item String manipulation
    \item Typecasting
    \item The \texttt{\&\&} and \texttt{||} operators
    \item Introduction to Object-Oriented Programming (OOP)
\end{itemize}

\subsection{Project Request}
We were introduced to a client project called \textbf{SpyTek}, where the task was to build a program capable of encrypting and decrypting email messages using the Caesar Cipher.

The client provided the following requirements:
\begin{itemize}
    \item The program should ask the user if they wish to encrypt or decrypt a message.
    \item It should then prompt the user to input the message and the key.
    \item The program will output the encrypted or decrypted message to the user.
\end{itemize}

\subsection{Plan of Action}
To solve this problem, we created a function that handles both encryption and decryption. The function takes the following parameters:
\begin{itemize}
    \item \texttt{text} - the message to be processed.
    \item \texttt{key} - the number of positions to shift in the alphabet.
    \item \texttt{mode} - whether to encrypt or decrypt.
\end{itemize}

\section{Caesar Cipher}
The Caesar Cipher is a substitution cipher that shifts each letter in the alphabet by a certain number of positions based on the key. Here's how it works:
\begin{itemize}
    \item Encryption shifts each letter forward by the key value.
    \item Decryption shifts each letter backward by the key value.
\end{itemize}

For example:
\begin{itemize}
    \item Shifting the letter 'a' by 5 results in 'f'.
    \item The reverse operation decrypts the message back to 'a'.
\end{itemize}
\begin{lstlisting}[language=java]
    import java.util.Scanner;
import java.util.StringTokenizer;

public class CaesarCypher1 {
    public static void main(String[] args) {
        final boolean DEBUG = false;
        Scanner input = new Scanner(System.in);
        //encryption or decryption
        System.out.println("What would you like to do?\n" +
                "1) encrypt\n" +
                "2) decrypt");
        int mode = input.nextInt();
        //key
        System.out.println("Enter the key size");
        int key = input.nextInt();
        //message
        input.nextLine();
        System.out.println("Enter the message");
        String msg = input.nextLine();
        //output result
        if (mode == 1){
            System.out.println(encrypt(msg, key));
        }
        else if(mode ==2) {
            System.out.println(decrypt(msg, key));
        }
        else {
            System.out.println("Choose 1 or 2");
        }
        if (DEBUG) System.out.println(mode+ " "+ key +" \n"+msg);
        
        input.close();
    }

    public static String encrypt(String msg, int key){
        StringTokenizer tokenizer = new StringTokenizer(msg.toLowerCase(), " ");
        String cypher = "";
        while (tokenizer.hasMoreTokens()){
            String value = tokenizer.nextToken();
            for (int i = 0; i< value.length(); i++){
                char character = value.charAt(i);
                int asciiValue = (int)value.charAt(i)+key;
                if(asciiValue > 122){
                    asciiValue-=26;
                }
                character = (char)asciiValue;
                cypher+=character;
            }
            cypher+=" ";
        }
        return cypher; // Change to message
    }
    public static String decrypt(String cypher, int key){
        StringTokenizer tokenizer = new StringTokenizer(cypher.toLowerCase(), " ");
        String text = "";
        while (tokenizer.hasMoreTokens()){
            String value = tokenizer.nextToken();
            for (int i = 0; i< value.length(); i++){
                char character = value.charAt(i);
                int asciiValue = (int)value.charAt(i)-key;
                if(asciiValue < 97){
                    asciiValue+=26;
                }
                character = (char)asciiValue;
                text+=character;
            }
            text+=" ";
        }
        return text; // Change to message
    }
}
\end{lstlisting}

\subsection{Key Information}
Before diving deeper into the implementation, we reviewed the following key topics:
\begin{itemize}
    \item \textbf{Encryption}: The process of encoding a message so that only the intended recipient can read it.
    \item \textbf{Decryption}: The process of converting the ciphertext back into readable text.
\end{itemize}

\section{String Manipulation}
String manipulation is key to the Caesar Cipher, and Java provides several tools to handle strings:
\begin{itemize}
    \item \texttt{substring(begin, end)} - Extracts part of a string.
    \item \texttt{toUpperCase()} - Converts a string to uppercase.
    \item \texttt{toLowerCase()} - Converts a string to lowercase.
    \item \texttt{replace(old, new)} - Replaces characters in a string.
    \item \texttt{charAt(index)} - Returns the character at the specified index.
    \item \texttt{length()} - Returns the length of the string.
\end{itemize}

\section{Typecasting}
In Java, typecasting allows you to convert between data types. There are two types of casting:
\begin{itemize}
    \item \textbf{Implicit casting}: Java automatically converts smaller data types to larger ones (e.g., from int to double).
    \item \textbf{Explicit casting}: Requires manual conversion using the target type in parentheses (e.g., \texttt{(int) value}).
\end{itemize}
\begin{lstlisting}
// Example of typecasting
public class TypeCastingExample {
    public static void main(String[] args) {
        // Implicit casting (int to double)
        int intValue = 10;
        double doubleValue = intValue;
        System.out.println("Implicit cast: " + doubleValue);

        // Explicit casting (double to int)
        double anotherDouble = 9.78;
        int anotherInt = (int) anotherDouble;
        System.out.println("Explicit cast: " + anotherInt);
    }
}
\end{lstlisting}
\newpage 
\section{ASCII Table}
We also used the ASCII table, which represents characters as numeric values. This allowed us to shift characters in the cipher by manipulating their ASCII values directly.
\begin{lstlisting}
// Example of using ASCII values in Caesar Cipher
public class ASCIICipher {
    public static void main(String[] args) {
        char c = 'a';
        int asciiValue = (int) c;
        System.out.println("ASCII value of 'a': " + asciiValue);
        
        // Shift character by 3 using ASCII
        char shiftedChar = (char) (c + 3);
        System.out.println("Shifted character: " + shiftedChar);
    }
}
\end{lstlisting}


\end{document}