\documentclass{article}
\usepackage{amsmath}
\usepackage{graphicx}
\usepackage{hyperref}
\usepackage{fancyhdr}
\usepackage{xcolor}
\usepackage{listings}

% Define custom colors
\definecolor{bg}{rgb}{0.64, 0.64, 0.82}
\definecolor{frame}{rgb}{0.59, 0.47, 0.71}
\definecolor{keyword}{rgb}{0.63, 0.14, 0.15}
\definecolor{comment}{rgb}{0.44, 0.5, 0.56}
\definecolor{string}{rgb}{0.12, 0.56, 0.18}

% Set custom listings options
\lstset{
    backgroundcolor=\color{bg},
    frame=single,
    rulecolor=\color{frame},
    basicstyle=\ttfamily\small,
    keywordstyle=\color{keyword}\bfseries,
    commentstyle=\color{comment},
    stringstyle=\color{string},
    showstringspaces=false,
    breaklines=true,
    xleftmargin=2mm,
    xrightmargin=2mm
}

\pagestyle{fancy}
\fancyhf{}
\fancyhead[L]{MAD 100 - Java Programming I - Fall 2024}
\fancyhead[R]{Instructor: Franco Iacobacci \thepage}
\title{Java Programming 1 - Week 5 Notes}
\author{Hia Al Saleh}
\date{October 3rd, 2024}

\begin{document}
\maketitle
\tableofcontents
\newpage

\section{Introduction to Functions}
A function is a block of code used to complete a specific task. Functions help to:
\begin{itemize}
    \item Simplify tasks and code.
    \item Remove duplicate code.
    \item Minimize the chance for errors.
\end{itemize}

In Java, a function is composed of the following elements:
\begin{itemize}
    \item Access Modifier (e.g., \texttt{public}, \texttt{private}).
    \item Non-Access Modifier (optional, e.g., \texttt{static}, \texttt{final}).
    \item Return Type (e.g., \texttt{int}, \texttt{boolean}, \texttt{void}).
    \item Function Name.
    \item Formal Parameters.
\end{itemize}

\section{Access Modifiers}
Access modifiers determine the visibility of functions. The common modifiers include:
\begin{itemize}
    \item \texttt{public}: The function can be accessed from anywhere.
    \item \texttt{private}: The function can only be accessed within the class.
    \item \texttt{protected}: The function can be accessed within the package and by subclasses.
\end{itemize}
Example:
\begin{lstlisting}[language=java]
public int calculateSum(int a, int b) {
    return a + b;
}
\end{lstlisting}

\section{Non-Access Modifiers}
Non-access modifiers add special features to functions:
\begin{itemize}
    \item \texttt{static}: Allows the function to be called without creating an object.
    \item \texttt{final}: Prevents overriding of the function.
    \item \texttt{abstract}: Creates a template for future functions (used in object-oriented programming).
    \item \texttt{synchronized}: Ensures only one thread uses the function at a time.
\end{itemize}

\section{Return Types}
The return type specifies the data type of the value returned by the function. Examples of return types include:
\begin{itemize}
    \item \texttt{int}, \texttt{double}, \texttt{boolean}, \texttt{String}, etc.
    \item \texttt{void}: Indicates that no value is returned.
\end{itemize}

Example:
\begin{lstlisting}[language=java]
public double findSquareRoot(double number) {
    return Math.sqrt(number);
}
\end{lstlisting}

\section{Naming Functions}
When naming functions:
\begin{itemize}
    \item Use camelCase for naming (e.g., \texttt{calculateTotal}).
    \item Choose meaningful names for clarity.
    \item Avoid starting names with \_ or \$.
\end{itemize}

\section{Formal Parameters}
Formal parameters allow you to pass data to a function when it is called. For example:
\begin{lstlisting}[language=java]
public int addNumbers(int num1, int num2) {
    return num1 + num2;
}
\end{lstlisting}
Here, \texttt{num1} and \texttt{num2} are formal parameters that the function uses.

\section{Code Audit}
We can improve code by modularizing repeated tasks into functions. For example, a \texttt{min()} function to return the smallest of two numbers:
\begin{lstlisting}
public static double min(double num1, double num2) {
    return (num1 < num2) ? num1 : num2;
}
\end{lstlisting}

Similarly, we can create a \texttt{max()} function to find the larger number, and an \texttt{equals()} function to check equality:
\begin{lstlisting}[language=java]
public static double max(double num1, double num2) {
    return (num1 > num2) ? num1 : num2;
}

public static boolean equals(double num1, double num2) {
    return num1 == num2;
}
\end{lstlisting}

\section{Building a Program (Example: HighLow)}
In this exercise, we will build a program that compares numbers entered by the user and keeps track of the highest and lowest numbers.

Steps:
\begin{enumerate}
    \item Create a new project in IntelliJ named \texttt{HighLow}.
    \item Create a new Java class file named \texttt{HighLow}.
    \item Write the \texttt{main()} method, which takes user input and stores it in a \texttt{Scanner} object.
    \item Use the \texttt{min()} and \texttt{max()} functions to update the highest and lowest numbers.
\end{enumerate}

Example:
\begin{lstlisting}[language=java]
public static void main(String[] args) {
    Scanner input = new Scanner(System.in);
    double num, highest, lowest;

    System.out.print("Enter a number: ");
    num = input.nextDouble();
    highest = lowest = num;

    // Continue accepting numbers until the user enters -1
    while (num != -1) {
        highest = max(highest, num);
        lowest = min(lowest, num);
        System.out.print("Enter a number: ");
        num = input.nextDouble();
    }

    System.out.println("Highest number: " + highest);
    System.out.println("Lowest number: " + lowest);
}
\end{lstlisting}

\section{Homework}
\begin{itemize}
    \item Read Chapter 6 of the textbook.
    \item Review the \texttt{min()}, \texttt{max()}, and \texttt{equals()} functions and ensure you understand their structure.
\end{itemize}

\end{document}