\documentclass{article}
\usepackage{amsmath}
\usepackage{graphicx}
\usepackage{hyperref}
\usepackage{fancyhdr}
\usepackage{xcolor}
\usepackage{listings}

% Define custom colors
\definecolor{bg}{rgb}{0.64, 0.64, 0.82}
\definecolor{frame}{rgb}{0.59, 0.47, 0.71}
\definecolor{keyword}{rgb}{0.63, 0.14, 0.15}
\definecolor{comment}{rgb}{0.44, 0.5, 0.56}
\definecolor{string}{rgb}{0.12, 0.56, 0.18}

% Set custom listings options
\lstset{
    backgroundcolor=\color{bg},
    frame=single,
    rulecolor=\color{frame},
    basicstyle=\ttfamily\small,
    keywordstyle=\color{keyword}\bfseries,
    commentstyle=\color{comment},
    stringstyle=\color{string},
    showstringspaces=false,
    breaklines=true,
    xleftmargin=2mm,
    xrightmargin=2mm
}

\pagestyle{fancy}
\fancyhf{}
\fancyhead[L]{MAD 100 - Java Programming I - Fall 2024}
\fancyhead[R]{Instructor: Franco Iacobacci \thepage}
\title{Java Programming 1 - Quiz Notes}
\author{Kensukeken}
\date{October 9th, 2024}

\begin{document}
\maketitle
\tableofcontents
\newpage


\section{Calculating Powers in Java}
In Java, you can calculate $2^8$ using the \texttt{Math.pow} method:
\begin{lstlisting}[language=Java]
double result = Math.pow(2, 8);
System.out.println(result);  // Outputs 256.0
\end{lstlisting}
Alternatively, using bitwise shift operators:
\begin{lstlisting}[language=Java]
int result = 1 << 8;
System.out.println(result);  // Outputs 256
\end{lstlisting}

\section{Math.abs Function}
The result of \texttt{Math.abs(7 - 9)} is:
\[
7 - 9 = -2
\]
Applying \texttt{Math.abs} gives $2$.

\section{Reading an Integer from User Input}
To read an integer from user input in Java, the following steps are required:
\begin{itemize}
    \item Import the \texttt{Scanner} class:
    \begin{lstlisting}[language=Java]
    import java.util.Scanner;
    \end{lstlisting}
    \item Create a \texttt{Scanner} object:
    \begin{lstlisting}[language=Java]
    Scanner input = new Scanner(System.in);
    \end{lstlisting}
    \item Use the \texttt{nextInt} method to read an integer:
    \begin{lstlisting}[language=Java]
    int value = input.nextInt();
    \end{lstlisting}
\end{itemize}

\section{Declaring Constants}
To declare a constant \texttt{HST} in Java, use the \texttt{final} keyword:
\begin{lstlisting}[language=Java]
final double HST = 13.0;
\end{lstlisting}

\section{Output of Floating Point Division}
The result of \texttt{System.out.println(4.0 / 2)} is:
\[
4.0 / 2 = 2.0
\]

\section{Modulo Operation}
The result of \texttt{System.out.println(7 \% 2)} is:
\[
7 \div 2 = 3 \quad \text{(quotient)}, \quad 1 \quad \text{(remainder)} 
\]
Thus, $7 \% 2 = 1$.

\section{Output Example}
To output the following:
\begin{quote}
Your numbers: 1 2 3 4 5 \\
Total 15
\end{quote}
The correct Java code is:
\begin{lstlisting}[language=Java]
System.out.println("Your numbers:");
System.out.print(1 + " ");
System.out.print(2 + " ");
System.out.print(3 + " ");
System.out.print(4 + " ");
System.out.print(5 + "\n");
System.out.print("Total ");
System.out.println(15);
\end{lstlisting}

\section{Math.floor Function}
The result of \texttt{Math.floor(3.9)} is $3.0$.

\section{Integer Division}
The result of \texttt{System.out.println(7 / 2)} is $3$ because Java performs integer division, truncating the decimal.

\section{Storing Decimal Numbers}
To store a decimal number like $7.5$ in Java, use the \texttt{double} type:
\begin{lstlisting}[language=Java]
double value = 7.5;
\end{lstlisting}

\section{Comparing Numbers}
The following comparisons are evaluated as:
\begin{itemize}
    \item $-2 < 1$: True
    \item $7 < -8$: False
    \item $-4 < -5$: False
    \item $3 < 3$: False
\end{itemize}

\section{For Loop Output}
For the following code:
\begin{lstlisting}[language=Java]
for (int i = 0; i <= 10; i++) {
    System.out.println(i);
}
\end{lstlisting}
The output is:
\[
0, 1, 2, 3, 4, 5, 6, 7, 8, 9, 10
\]

\section{Do-While Loop Output}
For the following code:
\begin{lstlisting}[language=Java]
int x = 10;
do {
    System.out.println(--x);
} while (x < 0);
\end{lstlisting}
The output is $9$ because the condition is checked after the first iteration.

\section{If-Else Statement}
Given the code:
\begin{lstlisting}[language=Java]
if (x >= 2) {
    System.out.println(">=2");
} else if (x >= 4) {
    System.out.println(">=4");
} else {
    System.out.println("<2");
}
\end{lstlisting}
If $x = 4$, the output will be \texttt{>=2}.

\section{Nested Loops Output}
For the following code:
\begin{lstlisting}[language=Java]
for (int i = 0; i < 3; i++) {
    for (int j = 0; j < 3; j++) {
        System.out.println(j + " - " + i);
    }
}
\end{lstlisting}
The output will be:
\[
0 - 0, 1 - 0, 2 - 0, 0 - 1, 1 - 1, 2 - 1, 0 - 2, 1 - 2, 2 - 2
\]

\section{Output of the while loop}
\begin{lstlisting}[language=Java]
int x = 0;
while (x < 10) {
    System.out.println(++x);
}
// Output:
1
2
3
4
5
6
7
8
9
10
\end{lstlisting}

\section{Switch Case Example}
For the following code:
\begin{lstlisting}[language=Java]
int x = 3;
switch(x) {
    case 1: System.out.println("one");
    case 2: System.out.println("two");
    case 3: System.out.println("three");
    case 4: System.out.println("four");
    default: System.out.println("other");
}
// Output:
three
four
other
\end{lstlisting}

\section{Generating a random integer between 1 and 10:}
\begin{lstlisting}[language=Java]
import java.util.Random;
Random random = new Random();
int randomValue = random.nextInt(10) + 1;
\end{lstlisting}

\section*{Break and Continue Statements}
The following code demonstrates the use of \texttt{break} and \texttt{continue}:
\begin{lstlisting}[language=Java]
for (int i = 0; i < 5; i++) {
    if (i == 2) {
        continue;
    }
    if (i == 4) {
        break;
    }
    System.out.println(i);
}
\end{lstlisting}
The output is:
\[
0, 1, 3
\]

\section*{Ternary Operator}
In Java, the ternary operator is used for conditional expressions:
\begin{lstlisting}[language=Java]
int result = (x > 0) ? x : -x;
\end{lstlisting}
If \texttt{x} is greater than 0, \texttt{result} is \texttt{x}; otherwise, \texttt{result} is \texttt{-x}.

\end{document}